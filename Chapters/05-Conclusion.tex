\chapter{Conclusion}
\label{cp:conclusion}
Using the Scanivalve DSA 3217 pressure transducers, we measured pressure from \num{43} taps on the upper and lower surfaces of an airfoil for a range of angle of attacks at a constant fluid velocity in the wind tunnel. We calculated the \gls{C_P}, \gls{C_L}, \gls{C_D}, \gls{C_M} from the Scanivalve data. The \gls{C_P} graphs show flow separation from the upper surface of the airfoil as the angle of attack increases. This separation causes a stall which is seen by the sudden increase of the \gls{C_D} in \autoref{fig:C_D vs. Angle of Attack.svg}, the maximum \gls{C_L} in \autoref{fig:C_L vs. Angle of Attack.svg}, and the minimum \gls{C_M} in \autoref{fig:C_M vs. Angle of Attack.svg} at the stall angle. Also from the \gls{C_P} graphs, the stagnation point can be seen moving away from the leading edge along the lower surface of the airfoil as the angle of attack increases.
