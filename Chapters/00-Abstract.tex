\thispagestyle{plain} % Page style without header and footer
% \pdfbookmark[1]{Resumo}{resumo} % Add entry to PDF
% \chapter*{Resumo} % Chapter* to appear without numeration
% \blindtext

% \keywordspt{Keyword A, Keyword B, Keyword C.}

% \blankpage

\pdfbookmark[1]{Abstract}{abstract} % Add entry to PDF
\chapter*{Abstract} % Chapter* to appear without numeration

Studying the aerodynamic characteristics of airfoils is a critical process in subscale testing. Using the low-speed wind tunnel at Iowa State University with several pressure transducers, we measured the pressure over a GA(W)-1 for various angles of attack. Based on our analysis and visualize of the data using a \acrfull{matlab} script, we found the stall angle was \qty{12}{\degree}. Additionally, we found that the stagnation point occurred at the tip of the airfoil for low \acrfull{aoa} and moved slightly to the right on the lower side of the airfoil for higher angles of attack. Our coefficient of pressure graphs also allowed us to identify when and to what degree the flow was separating on top of the airfoil when it underwent higher angles of attack. 

% \blankpage


