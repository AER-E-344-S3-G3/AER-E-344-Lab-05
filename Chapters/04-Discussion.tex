\chapter{Discussion}
\label{cp:discussion}

\autoref{fig:C_L vs. Angle of Attack.svg} shows the coefficient of lift at angles of attack of \qtyrange{-4}{16}{\degree}. The max \gls{C_L} is around an \acrshort{aoa} of \qty{12}{\degree}, which is the stall angle. \autoref{fig:C_D vs. Angle of Attack.svg} shows the lowest coefficient of drag is between \qtyrange{4}{6}{\degree}. The \gls{C_D} increases substantially after an \acrshort{aoa} of \qty{12}{\degree} which further shows that is the stall angle. The aerodynamic moment is negative for all angles of attack tested and decreases as the angle of attack increases (see \autoref{fig:C_M vs. Angle of Attack.svg}). The moment in the pitch-down direction increases as the airfoil \acrshort{aoa} increases. 

In \autoref{fig:C_p vs. -4° Angle of Attack.svg}, similar values of \gls{C_P} are seen on the upper and lower surface of the airfoil. The \acrshort{aoa} of \num{-4}\unit{\degree} is close to the zero-lift angle of attack as shown in \autoref{fig:C_L vs. Angle of Attack.svg}. The upper surface has a positive \gls{C_P} near the leading edge, while the lower surface has a negative \gls{C_P} resulting in a negative lift force.

\autoref{fig:C_P vs. 6° Angle of Attack.svg} shows that at an \acrshort{aoa} of \num{6}\unit{\degree}, the \gls{C_P} increases steadily from the leading to trailing edge. There is no flow separation over the airfoil. In contrast, \autoref{fig:C_P vs. 10° Angle of Attack.svg} shows that when the angle of attack nears \qty{10}{\degree}, the \gls{C_P} levels off at a normalized x coordinate of \num{0.6} and the flow starts to separate. In \autoref{fig:C_P vs. 16° Angle of Attack.svg}, the \gls{C_P} is almost completely constant along the upper surface showing the flow is completely separated at an \acrshort{aoa} of \qty{16}{\degree}.

The stagnation point is where fluid contacting the airfoil has a velocity of zero. At this point, the coefficient of pressure, \gls{C_P}, would be \num{1}, \textit{i.e.}, the pressure is equal to the total static pressure. From our data, at low angles of attack—\textit{e.g.}, \qtyrange{-4}{6}{\degree}—the \gls{C_P} closest to \num{1} appears at the normalized x coordinate of zero (see \autoref{fig:C_P vs. 0° Angle of Attack.svg}). However, at angles of attack greater than \num{6}\unit{\degree}, the \gls{C_P} closest to \num{1} moves to the right on the lower surface of the airfoil with a normalized $x$-coordinate of approximately \num{0.05} (see \autoref{fig:C_P vs. 8° Angle of Attack.svg}).

The motor frequency was set to \qty{15}{\hertz} for all angles of attack. Using calculations from Lab 2, the fluid velocity is around \qty{19.4}{\meter\per\second}. The Reynolds number at this velocity is approximately \num{1.3203e5}.